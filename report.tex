\documentclass[english,a4paper]{article}

\usepackage[T1]{fontenc}
\usepackage[latin9]{inputenc}
\usepackage[english]{babel}
\usepackage{lmodern}

\begin{document}

\title{Click-stream data analysis of web access logs}
\author{Riivo Kikas, Karl Potisepp}
\date{\today}
\maketitle

\begin{abstract}
In this paper, we will conduct a case study of mining frequent user access patterns.

\end{abstract}

\section{Introduction}
Understanding how users navigate and browse through web sites is important for many aspects, such as content recommendation, personalization and targeted advertising. Even more, it helps to evaluate usability and provide real world input to better organization of content and web site redesign. 

The aim of this report is to illustrate pattern mining from web logs with example of Computer Science department web access logs.  Motivation for this study was to find meaningful use-cases from log files and identify possible bottlenecks in the presentation layer and provide some input for user interface redesign. Main aim is to discover the most frequent sequence rules among the on the visitors dataset. The nature of pattern mining carried out was more of exploratory concentrating on frequent item sets and sequence mining.

Write here about sessions and data.

\section{Data preparation and session identification}
Input data for click-stream mining is in the form of standard Apache log files, specially in Common Log Format. This format provides information about from which IP address was request made, request data, page requested and returned HTTP status code.

\begin{figure}
\begin{verbatim}
192.168.1.245 - - [13/Sep/2009:04:07:56 +0300] "GET /ati/struktuur/tiiger HTTP/1.0" 200 19828
192.168.1.215 - - [13/Sep/2009:04:09:30 +0300] "GET / HTTP/1.0" 200 13511
192.168.1.215 - - [13/Sep/2009:04:12:02 +0300] "GET /103369 HTTP/1.0" 200 17367
192.168.1.215 - - [13/Sep/2009:04:14:40 +0300] "GET /varia/itinfo HTTP/1.0" 200 19411
\end{verbatim}
\caption{Example of log file}
\end{figure}

There are two important task to be carried out with log data before it can be used for analyzing. First, log files contain all requests made to server




We were interesting only in 
We added filtering to filter out. A line was discarded if:

\begin{itemize}
\item the URL pointed to a file that was an error page, someone's personal website or not a stand-alone web page (such as image files, JavaScript and stylesheets).
\item All other requests besides POST and GET, such as OPTIONS
\item the request originated from an IP address previously blacklisted for requesting {}``robots.txt''. We consider them as indexers.
\end{itemize}

In addition to that, URLs were also altered, removing traces of any query strings attributes. Some of the pages were generated dynamically and querystring contained  non human readable parameters that were different to users, but with eliminating several pages and subpages merged into on. One example of this students list pages. First pages states all majors whit url 15124 and some querystring. Now when user selects some major,  students listing is provided. The url does not changes, but querystring does. With eliminating querystring, we generalise user intentions in broader category, but still preservering meaningul information about the url.



After performing this procedure we were left with only data pertaining to human users' requests for individual pages. 

The next step is to group user requests made during on visit into session. Generally, this can be done by providing cookie to each user with some generated UUID and then logging each request and the cookie ID. But our goal was not to modify existing software stack and use the data already avaliabl. CLF formated log files do not have explicit session ID stored, so this needs to be derived. 
Simplemest method to extract sessions is to use timeout based metehod. We consider all requests from same IP address to be in same session, if the time between cliks is not more than $t=30$ seconds. If time between requests is larger than $t$, we consider these request into new session. 
It is the simplemest method available, but might not be the best. 


\section{Basic statistics about accesses}
We used batch of two log files.

The earliest request contained in the log files was dated 16th of August, 2009, the latest 8th of December on the same year. 

\section{Frequent item sets in log files}
First and foremost interesting question about web session on can asked, are there some pages that users tend to visit together during one session?

problems?,support, 

sven idea. reduce sparsity


\section{Frequent sequential patterns with timed access}

\section{Maximal forward references}

\section{Evolving patterns through time}

\section{Summary}

\end{document}
