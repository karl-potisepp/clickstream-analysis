\documentclass[english,12pt,a4paper]{article}
\usepackage[T1]{fontenc}
\usepackage[latin9]{inputenc}
\usepackage{setspace}
\usepackage[english]{babel}
\usepackage{lmodern}

\begin{document}

\title{Click-stream data analysis of web access logs}
\author{Riivo Kikas, Karl Potisepp}
\date{\today}
\maketitle

\begin{abstract}
for dm course we need to provide this

\end{abstract}

\section{Introduction}
Understanding how users navigate and browse through web sites is important for many aspects, such as content recommendation, personalization and targeted advertising. Even more, it helps to evaluate usability and provide real world input to better organization of content and web site redesign. 

Web site is now a representative
communication tool for teachers, students, and parents, etc. As
more people are visiting school Web sites, it is necessary to
provide right and up-to-date information by analyzing
requirements and characteristics of visitors

Exploratory Data Analysis


Our main aim is to discover the most frequent sequence rules among the 3
run first an exploratory analysis on the
visitors dataset. The available quantitative variables, which are somehow ancillary for

by providing detailed information on the patterns generated by users as they
navigate through a website hosted on that server. However, the size and nature of



The aim of the project was to analyse the access log files of the faculty's web site (http://www.math.ut.ee) Apache server in order to find any meaningful use-cases and to identify possible bottlenecksin the presentation layer. In order to achieve this, we used a Python-based Apache log parser and experimented with a number of data mining tools(such as Apriori and FPgrowth algorithms). More detailed results and description of the work done follow.






\section{Log file structure and session identification , extraction}

The raw data was in the form of several .log files containing anonymized 
IP address, time of request, type of request (either GET or OPTIONS),
the page requested, and the HTTP return code for it. The earliest
request contained in the log files was dated 16th of August, 2009,
the latest 8th of December on the same year. Sample lines from one
of the log files can be seen below:
\begin{quotation}
\begin{singlespace}
\texttt{1 - - {[}13/Sep/2009:06:14:58 +0300{]} \textquotedbl{}OPTIONS
/ HTTP/1.1\textquotedbl{} 200 13511 }

\texttt{7 - - {[}13/Sep/2009:06:15:22 +0300{]} \textquotedbl{}GET
/inimesed/Instituudid HTTP/1.1\textquotedbl{} 200 20741 }

\texttt{7 - - {[}13/Sep/2009:06:15:22 +0300{]} \textquotedbl{}GET
/varia/killustik HTTP/1.1\textquotedbl{} 200 33579}\end{singlespace}

\end{quotation}
Since our task was to try and find possible ways of improving the
math.ut.ee experience for users, we decided to filter out requests
that did not contain any real information. For this we first used
an open source Apache log file parser written in Python and read the
log files to memory. A line was discarded if:
\begin{itemize}
\item ...the URL pointed to a file that was an error page, someone's personal
website or not a stand-alone web page (such as image files, JavaScript
and stylesheets).
\item ...the line contained OPTIONS instead of GET (i.e. was not a page
request).
\item ...the request originated from an IP address previously blacklisted
for requesting {}``robots.txt''.
\end{itemize}
In addition to that, URLs were also altered, removing traces of any
HTTP GET attributes. After performing this procedure we were left
with only data pertaining to human users' requests for individual
pages. The next step was to group requests into sessions to create
a dataset for running frequent itemset detection algorithms. To achieve
this, we grouped all URLs by first looking at the originating IP address,
then the time interval between consecutive requests. On the side we
also calculated the probable average amount of time spent on one page
by finding the amount of seconds between the original request and
the next request from the same IP within the same session.


\section{Basic statistics about accesses}
\section{Frequent item sets in log files}
First and foremost interesting question about web session on can asked, are there some pages that users tend to visit together during one session?


\section{Frequent sequential patterns with timed access}
\section{Maximal forward references}
\section{Evolving patterns through time}
\section{Summary}

\end{document}
